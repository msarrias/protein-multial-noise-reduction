\documentclass[12pt]{article}
\usepackage{amsmath,amsfonts,amsthm,amssymb}
\usepackage[english]{babel}
\usepackage[T1]{fontenc}
\usepackage{graphicx,epstopdf,subfig}
\usepackage{color,url}
\usepackage{mathtools,paralist,algorithm,algorithmicx}
\usepackage{fancyvrb}
\usepackage{hyperref}
\usepackage[procnames]{listings}


\begin{document}

\begin{titlepage}
\title{\vspace{60mm}Reducing noise in protein multialignments}
\author{Marina Herrera Sarrias}
\date{January 6, 2019}

\clearpage\maketitle
\thispagestyle{empty}
\end{titlepage}

\pagenumbering{roman}
\section*{Abstract}

\href{www.wikibooks.org}{Wikibooks home}
In this research a noise-reduction method is implemented in protein multialigments to evaluate its impact on phylogenic inference.The data used in this project is a reduced data set from the  original data used by the creators of TrimAI. To test the data the program fastprot was used to obtain the distance matrices and fnj to infer a phylogenic tree for each alignment. Dendropy has been used to measure the symmetric distance between the inferred tree and the reference tree. %This study focuses in determining whether the noise reduction  

\addcontentsline{toc}{section}{Abstract}
\clearpage
	
\tableofcontents
\clearpage

\pagenumbering{arabic}
\section{Introduction}

The implemented noise-reduction method evaluates a multialigment column as noisy if there are more than $50\%$ indels, at least $50\%$ of the amino acids are unique and non of the amino acids appear more than twice. Each column takes the form of a sequence of amino acids given a position of the aligned sequences.

The reduced data set used in this project is composed by six different directories, each directory contains a reference tree and 300 alignments created by evolving sequences along the reference tree. Each pair of directories present an average amount of mutations of 0.5, 1.0 and 2.0 per sequence site. Also, per each mutation rate there is one symmetric reference tree and one asymmetric reference tree.

For validating the effect of reducing reducing 

\section{Results and Discussion}
\section{Methods and Materials}
\clearpage

%\begin{figure}[h]
%\centering
%	\includegraphics[width=.45\textwidth]{NAest} 
%	\caption{Nelson-Aalen estimates for cummulative hazard rate function with 95\% confidence interval  for group 1 (groups = 0) and group 2 (groups = 1).}
%	\label{fig:na}
%\end{figure}

\end{document}
